% !TeX root = ../main.tex
% Add the above to each chapter to make compiling the PDF easier in some editors.

\chapter{Summary of Notation}

\begin{fullwidth}
We follow these general rules: \begin{itemize}[noitemsep]
    % \item uppercase italic for constants $N$
    \item lowercase italic for indices $i$ and scalar variables $a$
    % \item lowercase italic bold for vectors $\vx$
    \item uppercase italic bold for matrices $\mM$
    % \item uppercase italic for random variables $X$
    % \item uppercase bold for random vectors $\rX$
    \item uppercase italic for sets $A$
    % \item uppercase calligraphy for spaces (usually infinite sets) $\spA$
\end{itemize}

\emptyparagraph\begin{tabular}{p{2cm}l}
    $\defeq$ & equality by definition \\
    $f : A \to : B$ & function $f$ from elements of set $A$ to elements of set $B$ \\
    $\N$ & set of natural numbers $\{1, 2, \dots\}$ \\
    $\NZ$ & set of natural numbers, including $0$, $\N \cup \{0\}$ \\
    $[n]$ & set of natural numbers from $1$ to $n$, $\{1, 2, \dots, n-1, n\}$ \\
    $\Z$ & set of integers $\{\dots, -2, 1, 0, 1, 2, \dots\}$ \\
    $\R$ & set of real numbers \\
    $\C = \R^2$ & set of complex numbers \\
\end{tabular}

\section*{\smallcaps{Groups}}
\emptyparagraph\begin{tabular}{p{2cm}l}
    $\GL{n}{K}$ & \emph{general linear group} over invertible linear maps $\mA \in K^{n \times n}$, $\determ{\mA} \neq 0$ \\
    $\SL{n}{K}$ & \emph{special linear group} over volume-preserving linear maps $\mA \in K^{n \times n}$, $\determ{\mA} = 1$ \\
    $S_n$ & \emph{symmetric group} over bijections on $[n]$ (so-called permutations) \\
\end{tabular}
\end{fullwidth}
