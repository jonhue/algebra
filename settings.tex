% !TeX root = main.tex
% Add the above to each chapter to make compiling the PDF easier in some editors.

\setcounter{secnumdepth}{2}
\setcounter{tocdepth}{1}

%--------Packages--------
\usepackage{tikz}
\usetikzlibrary{calc, arrows}
\usetikzlibrary{shapes}
\usetikzlibrary{positioning}
\usepackage{booktabs}
\usepackage{tabularx}
\usepackage{makecell}
\usepackage{amsmath,amsfonts,amsthm,amssymb,mathrsfs,bbm,mathtools,nicefrac}
\usepackage[shortlabels]{enumitem}
\usepackage{cleveref}
\usepackage[italic]{derivative}
\usepackage{xparse}
\usepackage{upgreek}

% %--------Literature--------
% \bibliographystyle{plainnat}
% % \usepackage{xargs}
% % \renewcommandx{\cite}[3][1={0pt},2={}]{\sidenote[][#1]{\fullcite[#2]{#3}}}

%--------Graphics/Images--------
\usepackage{graphicx}
% \setkeys{Gin}{width=\linewidth,totalheight=\textheight,keepaspectratio}
\graphicspath{{./figures/}}

% %--------Index--------
\usepackage{imakeidx}
\makeindex[intoc]
\indexsetup{headers={\indexname}{\indexname}}

%--------Enumerations--------
\setlist{itemsep=0pt,topsep=2pt}

%--------Theorem Environments--------
\newtheorem{thm}{Theorem}[chapter]
\newtheorem{cor}[thm]{Corollary}
\newtheorem{lem}[thm]{Lemma}

\theoremstyle{definition}
\newtheorem{defn}[thm]{Definition}

\theoremstyle{remark}
\newtheorem{rmk}[thm]{Remark}

\numberwithin{equation}{chapter}

%--------Pseudocode--------
\usepackage{xcolor,amsmath}
\usepackage[linesnumbered,ruled,vlined,algochapter]{algorithm2e}
\DontPrintSemicolon
\makeatletter
    \let\c@algocf\c@thm
\makeatother
\crefname{algocf}{alg.}{algs.}
\Crefname{algocf}{Algorithm}{Algorithms}
\SetAlgoSkip{bigskip}

% Define pseudocode formatting
\renewcommand{\KwSty}[1]{\textnormal{\textcolor{blue!90!black}{\ttfamily\bfseries #1}}\unskip}
\renewcommand{\ArgSty}[1]{\textnormal{\ttfamily #1}\unskip}
\SetKwComment{Comment}{\color{green!50!black}// }{}
\renewcommand{\CommentSty}[1]{\textnormal{\ttfamily\color{green!50!black}#1}\unskip}
\newcommand{\assign}{\leftarrow}
\newcommand{\var}{\texttt}
\newcommand{\FuncCall}[2]{\texttt{\bfseries #1(#2)}}
\SetKwProg{Function}{function}{}{}
\renewcommand{\ProgSty}[1]{\texttt{\bfseries #1}}

% %--------Colors-------------
% \definecolor{baby-blue}{cmyk}{0.86,0.33,0.0,0.16,1.0}
\definecolor{blue}{RGB}{02,106,253}
\definecolor{red}{RGB}{245,51,30}
\definecolor{green}{RGB}{96,189,69}
\def\b{\textcolor{blue}}
\def\r{\textcolor{red}}

%--------Colored Theorem-------------
\usepackage[most]{tcolorbox}
\tcbuselibrary{theorems}
\newtcbtheorem
  [use counter*=thm,number within=chapter,crefname={example}{examples},Crefname={Example}{Examples}]%
  {ex}
  {\textbf{Example}}
  {%
%   boxrule=1pt,
%   toprule=1pt,bottomrule=1pt,
    before skip=10pt,after skip=10pt,
    left=0.2cm,right=0.2cm,top=0.2cm,
%   titlerule=0.5em,
%   toptitle=0.1cm,
%   bottomtitle=-0.1cm,
    breakable,
    sharp corners,
    before upper=\setlength{\parskip}{8pt},
    % colback=white,
    % coltitle=black,
    % colframe=blue!100,
    fonttitle=\bfseries,
  }% options
  {ex}% prefix

%--------Margin Tags-------------
\let\marginnote\relax
\usepackage{marginnote}
\NewDocumentCommand{\margintag}{mO{0pt}}{\checkoddpage\ifoddpage{\marginnote{\footnotesize #1}[#2]}\else{\reversemarginpar\marginnote{\footnotesize #1}[#2]}\fi}
\NewDocumentCommand{\marginfootnote}{om}{\footnotemark\margintag{\textsuperscript{\tiny\arabic{footnote}} #2}[#1]}

%--------Equation numbers in text-------------
\makeatletter
\NewDocumentCommand{\embeq}{m}{%
  \leavevmode\hfill\refstepcounter{equation}\textup{\tagform@{\theequation}}\label{#1}%
}
\makeatother

%--------Equation numbers in algorithms-------------
\makeatletter
\NewDocumentCommand{\algeq}{m}{%
  \leavevmode\Comment*[r]{\refstepcounter{equation}\textup{\tagform@{\theequation}}\label{#1}}%
}
\makeatother

\usepackage{etoolbox}
\makeatletter
% Remove right hand margin in algorithm
\patchcmd{\@algocf@start}% <cmd>
  {-1.5em}% <search>
  {0pt}% <replace>
  {}{}% <success><failure>
\makeatother

%--------Allow page breaks in align-------------
\allowdisplaybreaks
